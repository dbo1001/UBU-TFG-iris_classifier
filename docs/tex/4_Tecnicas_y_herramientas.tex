\capitulo{4}{Técnicas y herramientas}

\section{Gestión del proyecto}
\subsection{Scrum}
Se trata de una metodología ágil para el desarrollo \emph{software}. Se basa en el desarrollo de pequeñas tareas que se revisaran e irán encrementando en un plazo establecido (\emph{sprints)} ~\cite{wiki:scrum}.
\subsection{Github}
Es una plataforma para el alojamiento de proyectos usando el sistema de control de versiones de \emph{git} .
\subsection{Git}
Es un sistema de control de versiones que se encarga de llevar un registro de los cambios en los proyectos y que permite coordinar el trabajo cuando el proyecto es compartido ~\cite{wiki:git}.

\section{Herramientas}
\subsection{Python}
Es un lenguaje de programación interpretado, dinámico y orientado a objetos. Se trata de uno de los lenguajes de programación más populares actualmente, entre otras cosas por su corta curva de aprendizaje y su versatilidad lo hace un lenguaje idóneo para el \emph{data science} ~\cite{wiki:python}.

\subsection{Anaconda}
Es una suite multiplataforma creado específicamente para el \emph{data science} que incorpora \emph{Jupyter Notebooks} los cuales permiten crear y compartir documentos que contiene código, ecuaciones y facilita la experimentación con \emph{Machine Learning}.
\subsection{JupyterLab}
Se considera la siguiente generación de interfaces basadas en la web para el projecto \emph{Jupyter}. JupyterLab permite trabajar con Jupyter notebooks, editores de texto y terminales. Se eligió esta herramienta como alternativa a los noteboks tradicionales por ser más cómoda.

\section{Bibliotecas de Python}
\subsection{Numpy}
Biblioteca que soporta el trabajo con vectores y matrices de gran tamaño que incluye una gran colección de funciones matemáticas listas par su uso.
\subsection{Pandas}
Se trata de una herramienta \emph{open source} destinada al análisis y manipulación de datos rápida, potente, flexible y fácil de usar.

\subsection{OpenCV}
Es la biblioteca \emph{open source} multipataforma para la visión por computadora por excelencia. Está publicada bajo la licencia BSD 3-Clause que permite que sea usada libremente para propósitos de investigación y comerciales. El hecho de disponga de un buena documentación que se actualiza constantemente es uno de los factores que determina su popularidad ~\cite{wiki:opencv}.

\subsection{Scikit Image}
Biblioteca con una serie de algoritmos para el procesamiento de imágenes.
\subsection{Scikit Learn}
Biblioteca de \emph{Machine Learning} para \emph{Python}. Incluye algoritmos de clasificación, regresión y \emph{clustering} ~\cite{wiki:sklearn}.

\subsection{Keras}
Biblioteca de \emph{deep learning} diseñada para la experimentación y el desarrollo con modelos de aprendizaje profundo compatible con los \emph{frameworks} de \emph{TensorFlow}, \emph{MCTK} o \emph{Theano} ~\cite{wiki:keras}.

\subsection{TkInter}
Binding de la biblioteca gráfica de \emph{Tcl/Tk} considerda el estándar de interfaz gráfica para Python.

\section{Documentación}
\subsection{\LaTeX}
Es un sistema de composición de textos orientado a la creación de documentos que presenten una alta calidas tipográfica ~\cite{wiki:latex}. Su amplia gama de posibidades lo hace una herrmienta muy usada en la redacciónde artículos y libros científicos que incluyan expreiones matemáticas entre otras cosas.
\subsection{Texmaker}
Es un editor para la redacción de documentos con \LaTeX. Para su funcionamiento es necesario la previa instalación de \emph{Tex}. Al final se cambió por \emph{Overleaf} por problemas continuos a la hora de compilar los ficheros .tex.
\subsection{Overleaf}
Es un editor de texto online para la redacción de documentos con \LaTeX. El hecho de que sea online hace que no sea necesaria ninguna instalación y permite entre otras cosas la colaboración en compartida en tiempo real , control de versiones y trabajar desde cualquier lugar.