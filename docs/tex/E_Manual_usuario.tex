\apendice{Documentación de usuario}

\section{Introducción}
Se explicará brevemente los requisitos que un usuario normal debe llevar a cabo para ejecutar el proyecto.
\section{Requisitos de usuarios}
El usuario deberá contar con computador independientemente de si es \emph{Windows} o \emph{Linux} (en mac OS se tiene dudas por ser el único sistema operativo al que no se ha tenido acceso) con \emph{Python} y \emph{Anaconda} instalado.
\section{Instalación}
Seguir los mismos pasos de la sección \ref{pasos}.

\section{Manual del usuario}
Las tareas que permite la aplicación son:
\subsection{Cargar muestra}
Pasos:
\begin{itemize}
    \item Pulsar en el botón \emph{Cargar muestra}.
    \item Seleccionar la muestra deseada.
\end{itemize}

\imagen{app1}{Pantalla inicial con única opción de cargar una imagen.}
\imagen{app2}{Ventana de carga.}
\imagen{app3}{Imagen cargado y  disponible el botón de clasificar.}
\subsection{Clasificar}
Pulsamos en el botón \emph{Clasificar}.
\imagen{app4}{Resultado de la clasificación.}

\subsection{Segmentar}
Pulsamos en el botón \emph{Segmentar}.
\imagen{app5}{Muestra segmentada.}

\subsection{Mostrar bordes límbico y pupilar}
Pulsamos en el botón \emph{Coordenadas}.
\imagen{app6}{Muestra con los bordes dibujados.}

\subsection{Normalizar}
Pulsamos en el botón \emph{Normalizar}.
\imagen{app7}{Muestra normalizada.}

Es posible que dependiendo del \emph{hardware} del equipo con el que se trabaje, aparezca el típico aviso de \emph{No responde} después de pulsar el botón \emph{Clasificar}, hacer caso omiso de dicho aviso y en ningún momento cancelar la ejecución.