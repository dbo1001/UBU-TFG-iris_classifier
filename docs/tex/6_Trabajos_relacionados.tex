\capitulo{6}{Trabajos relacionados}

\section{Artículos}
\subsection{\emph{Diagnóstico de la diabetes mediante el análisis del iris}}
Se trata de un proyecto de investigación realizado por los alumnos del bachillerato de excelencia Raúl Santamaría, Jana Bisabarros y Estrella Higuera en el año 2017.

Su estudio se centra en averiguar si los patrones del iris pueden delatar indicios de diabetes en pacientes. Aunque el objetivo final es distinto al estudiado en este proyecto, puede observarse ciertas etapas comunes a ambos proyectos, como pueden ser la fase de segmentación y normalización. Las metodologías para realizar dichas etapas son distintas pero sirvieron de inspiración  al principio del desarrollo de este proyecto cuando se tenía dudas de como empezar.

\subsection{\emph{Image Understanding for Iris Biometrics: A Survey}}
Es un estudio por parte de Kevin Bowyer, Karen Hollingsworth y Patrick Flynn, en la que se profundiza en el estado del arte de los métodos existentes para el uso del iris como biometría. Destacar que se refiere a los métodos tradicionales, es decir, a los matemáticos, y no a los nuevos que incluyen técnicas de \emph{Machine Learning}.


\subsection{\emph{A multi-biometric iris recognition system based on a deep learning approach}}

Alaa S. Al-Waisy, Rami Qahwaji, Stanley Ipson, Shumoos Al-Fahsawi y Tarek A.M. Nagen son los autores de este \emph{paper} en el que se propone el uso de modelos de \emph{deep learning} para el reconocimiento del iris, cabe destacar que no es el primero en el que se propone este nuevo enfoque.

Desarrollan un modelo de \emph{deep learning} propio llamado \emph{IrisConvNet} para el entrenamiento con 3 bases de datos de iris distintas obteniendo un sistema de reconocimiento con una tasa de acierto del 100\% en menos de un segundo.
\imagen{convnet}{Arquitectura de \emph{IrisConvNet}}

