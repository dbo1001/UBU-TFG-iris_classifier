\capitulo{2}{Objetivos del proyecto}

\section{Objetivos generales}

\begin{itemize}
	\item Investigar sobre las técnica existentes en el ámbito del reconocimiento de individuos mediante el iris.
	\item Escoger uno de las técnicas y profundizar en ella con el fin de obtener un modelo funcional.
	\item Facilitar la comprensión de los resultados obtenidos a cualquiera con ningún conocimiento del tema que se trata.
	\item Desarrollar un aplicación gráfica de escritorio que permita probar la funcionalidad estudiada.
\end{itemize}

\section{Objetivos técnicos}

\begin{itemize}
	\item Aprender a usar bibliotecas populares de \emph{Machine Learning} cómo \emph{Scikit-learn} o \emph{Keras}
	\item Usar modelos preentrenados de \emph{Deep learning} para la extracción de los patrones del iris.
	\item Desarrollar una aplicación de escritorio mediante \emph{Kivy}.
	\item Usar GitHub para alojar el proyecto y realizar su seguimiento y Git como sistema de control de versiones.
	\item Implementar la metodología ágil estudiada en la carrera, Scrum.
\end{itemize}

\section{Objetivos personales}

\begin{itemize}
	\item Desarrollar un proyecto fuera de los propuestos por la Universidad.
	\item Comprender el funcionamiento de un sistema de reconocimiento e intentar realizar alguna aportación personal en dicho campo.
	\item Aplicar y ampliar los conocimientos adquiridos durante la carrera.
	\item Adentrarme en el mundo de la inteligencia artificial de manera autodidacta o mediante cursos online como Udemy y Coursera.
	\item Adentrarme en el desarrollo de aplicaciones de escritorio.
\end{itemize}